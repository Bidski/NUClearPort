% Initialisation
\documentclass[english,12pt]{scrartcl}

\usepackage[]{babel}
% Input is utf8
\usepackage[utf8]{inputenc}
% Enables headers and footers
\usepackage[]{scrpage2}
% Lets us colour table cells
\usepackage[table]{xcolor}
% Allows todo list and todos
\usepackage[]{todonotes}
% Makes links in contents hyperlinked
\usepackage{hyperref}
% Make references appear in our table of contents
\usepackage[nottoc,numbib]{tocbibind}
% Allows us to put landscape sections of the document
\usepackage{pdflscape} % \usepackage{lscape} %Use escape for printing (doesn't rotate the pdf page)
% Provides a glossary
\usepackage[toc]{glossaries}

% Gives us pretty diagrams
\usepackage{tikz}
\usetikzlibrary{calc,fit,positioning,chains,decorations.pathreplacing,shapes,backgrounds}

% Document Title and Author
\title{NUClear Design Document}
\author{2013 Final Year Project}

% Header and Footer
\pagestyle{scrheadings}
\ihead{\today}
\chead{}
\ohead{NUClear Design Document}
\ifoot{}
\cfoot{}
\ofoot{\pagemark}

% Requirements custom commands
\newcommand{\requirement}[1]{\textit{#1}}

% Skip line rather then indent paragraphs
\setlength{\parindent}{0.0in}
\setlength{\parskip}{0.1in}

% Start of document
\begin{document}
	\maketitle
	\vfill
	{\large
		\begin{description}
			\item [Status:] Draft 1
			\item [Version:] 0.1
		\end{description}}

	\clearpage
	\tableofcontents

	\section{Document Notes}
		\begin{tabular}{ p{0.1\textwidth} | p{0.6\textwidth} | p{0.3\textwidth} }
			\textbf{Version} & \textbf{Changes} & \textbf{Author} \\
			\hline

			0.1 &
			Initial Template &
			Jake Woods \\
			\hline
		\end{tabular}
		
	\clearpage
		
	\section{Introduction}
        NUClear is a C++ library that enables developers to write modular systems that communicate through messages. 
        NUClear provides the framework for software modules to send and recieve messages while remaining completely unaware of their origin and destintaion.
        It was originally designed to replace a robotics system that utilized a shared global memory store for communication and excels in replacing systems
        that communicate through a globally shared "god-object".

        NUClear leverages modern C++ features to provide both speed and simplicity. 
        The public API contains only three functions and much of the computation is performed at compile-time to avoid incurring a runtime performance penalty.

        NUClear is designed to run well on systems that do not have powerful processors. As such we have opted to use binary message passing instead of the more
        traditional serialization based (XML, JSON, etc...) message passing due to the heavy performance penalty associated with text-based serialization/deserialization.

    \section{Features}
        \subsection{Transparent Multithreading}
            \todo[inline] {Write about feature}
        
        \subsection{Compile-Time Message Routing}
            \todo[inline] {Write about feature}

        \subsection{Smart Timers}
            \todo[inline] {Write about feature. No wasted CPU cycles using a wait-loop}

        \subsection{Statistics and Logging}
            \todo[inline] {Write about stats and logging and how we solve the problem of multithreaded logging}

    \section{Component Overview - Power Plant}
        \todo[inline] {Expand this a lot}
        The PowerPlant has three main responsibilities which are split up into discreet classes.

        \subsection{ReactorMaster}
            \todo[inline] {Write about ReactorMaster}

        \subsection{ThreadMaster}
            \todo[inline] {Write about ThreadMaster}

        \subsection{CacheMaster}
            \todo[inline] {Write about CacheMaster}

    \section{Component Overview - Reactor}
        \todo[inline] {Write about Reactor and consider if On/Emit/Log are the appropriate subsections}

        \subsection{On}

        \subsection{Emit}

        \subsection{Log}

    \section{Component Overview - Reaction}
        \todo[inline] {Explain how the concept of a Reaction fits in here}

    \section{Component Overview - Networking}
        \todo[inline] {Talk about Networking as an Extension here}

	\bibliographystyle{plain}
	\bibliography{references}
	
	\printglossaries
\end{document}
